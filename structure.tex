% DOCUMENT MARGINS
\usepackage{geometry}
\geometry{
paper=a4paper, % SET PAPER TYPE
landscape,     % SET LANDSCAPE
twocolumn,     % SET TWO COLUMN
includehead,   % ALLOW HEADER
includefoot,   % ALLOW FOOTER
nomarginpar,   % NO MARGIN NOTES
% MARGINS
top=0.50in,
bottom=0.40in,
left=0.50in,
right=0.50in,
footskip=0.45in
}

% HEADER AND FOOTER
\usepackage{titling}
\usepackage{fancyhdr}
\pagestyle{fancy}
\fancyhf{}
\renewcommand{\headrulewidth}{0.4pt}
\renewcommand{\footrulewidth}{0.3pt}
\fancyhead[R]{\leftmark}
\fancyhead[L]{\thetitle}
\fancyfoot[C]{\thepage} % PRINT PAGE NUMBERS IN THE FOOTER
\fancyfoot[L]{\theauthor}
\fancyfoot[R]{\thedate}

% MATH
\usepackage{pifont} % \ding{114}
\usepackage{amsthm}
\usepackage{amsfonts}
\usepackage[dvipsnames]{xcolor}
\newcommand{\notesym}{\ding{114}}
\newcommand{\notecolor}{\color{MidnightBlue}}

% DEFINITION ENVIRONMENT
\newtheoremstyle{note}% 〈name〉
{3pt}% 〈Space above〉1
{0.1in}% 〈Space below 〉1
{}% 〈Body font〉
{}% 〈Indent amount〉2
{\bfseries}% 〈Theorem head font〉
{}% 〈Punctuation after theorem head 〉
{.5em}% 〈Space after theorem head 〉3
{\notecolor\notesym\thmnote{ #3 }\color{} --}% 〈Theorem head spec (can be left empty, meaning ‘normal’ )〉
\theoremstyle{note}
\newtheorem*{note}{}
